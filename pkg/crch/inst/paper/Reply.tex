\documentclass[12pt,a4paper]{article}

\setlength{\parskip}{0.5ex plus0.1ex minus0.1ex}
\setlength{\parindent}{0em}

\begin{document}

\title{Rejoinder to: "Heteroscedastic Censored and Truncated Regression with crch"}
\maketitle

We thank the reviewer for the valuable suggestions and comments. The comments
and our corresponding replies can be found below:

\textit{I enjoyed very much reading the manuscript, because it is well written
and describes an interesting and relevant R package: ``crch'' for
heteroscedastic censored and truncated regression. The package is also well
prepared (e.g. the source code is well written and commented so that it is
quite easy to read for humans) and well documented.}

Thank you very much for the positive feedback, much appreciated.

\medskip

\textit{In the following, I will give a few minor recommendations for improvement:}
\begin{enumerate}
  \item \textit{On page 1, last line, the term "log ($g(\sigma) =
  \log(\sigma)$)" is easily misunderstood as "$\log(g(\sigma) =
  \log(\sigma))$," which does not really make sense.  I suggest to write
  instead something like "logarithm, i.e. $g(\sigma) = \log(\sigma)$."} 
  \item \textit{In the last line of page 2 and in the following lines on page
  3, several tilde symbols (\textasciitilde) are missing. They can be obtained,
  e.g., by the \LaTeX command \textbackslash textasciitilde.} 
  \item \textit{On
  page 3, in the third from last paragraph, the last sentence should be
  terminated with a full stop (.).} 
  \item \textit{On page 3, in the last paragraph, I suggest to mention the
  software environment (R), for which the packages gamlss, gamlss.cens, and
  gamlss.tr are written for.}
\end{enumerate}

Thanks for spotting these. We have implemented all of these modifications.

\medskip

\begin{enumerate}
  \setcounter{enumi}{4}
  \item \textit{I have some concern regarding the example with the two-part
  model: as far as I can see, this model does not take a into account a
  possible correlation between the error term of the logit (selection) model
  and the error term of the truncated (outcome) model, although it is quite
  likely that these two error terms are correlated. In order to allow for
  different processes that drive the occurrence of precipitation and the
  precipitation amount, a heteroscedastic (Heckman-type) sample selection model
  is probably more appropriate.}
\end{enumerate}

This is an interesting comment and it is indeed conceivable that such correlations
between the error terms exist.

However, the usual approach of the Heckman selection
model is not directly feasible (even when adding support for heteroskedasticity).
The reason is that the distribution of the dependent variable in the outcome
equation is an uncensored/untruncated normal distribution in the Heckman case. For example,
in the classic wage equation model the response in the outcome equation is the
log-wage for which a normal distribution is a useful model and a semi-logarithmic
or log-log-specification of the regression (rather than linear in levels) is
of economic interest. In contrast, for precipitation it does not make sense
to model log-precipitation if the main regressor is the mean ensemble precipitation.
And considering the log on the right hand side for the mean ensemble precipitation
as well would introduce the problem that the ensemble mean may well have been zero
although the actual observed precipitation was positive.

As an alternative we considered a double hurdle model (using the R package
mhurdle) to account for the zero truncation of the ``selected'' outcomes.
However, this yielded only a very small and non-significant correlation.
Therefore, we did not pursue this idea further.

Moreover, even if there were some correlations, we feel that a clean disentanglement
of the processes is less important in our application compared to, say, wage
equations in labor market econometrics. Typically, there are no interventions
that try to change the selection probabilities or outcome expections in
precipitation in a certain area (in contrast to labor market programs etc.),
so the goal is ``just'' to obtain good/better forecasts than standard models.
As this is the case for the two-part model compared to the censored model,
it is worth considering.

Nevertheless, we will keep your idea in mind. Even if it did not improve the
forecasts in this application, it may be useful for another area or another
atmospheric quantity (e.g., wind speed).

But as all of this is somewhat beyond the scope of the R package crch and the
accompanying manuscript, we just added a short sentence to indicate that
we are mainly interested in testing the tobit model assumptions:
``In the censored model the occurrence of precipitation and precipitation amount
are assumed to be driven by the same process. To test this assumption we compare
the censored ..." 
\end{document}
